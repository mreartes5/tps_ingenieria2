\section{User Stories}

Las user stories del Backlog son:
\begin{enumerate}
\item Como dueño quiero que el celular del asegurado se comunique por bluetooth con el dispositivo del auto para que la aplicación reconozca que esta en el auto 
asegurado. 

Story points: 1.

\item Como dueño quiero que la aplicación esté corriendo en background para poder conectarse con el dispositivo en el auto asegurado. 

Story points: 
1.

\item Como dueño quiero que la aplicación registre la información de la posición actual y velocidad para guardar los datos que servirán para los cálculos del 
scoring. 

Story Points: 2

\item Como dueño quiero que la aplicación registre el total de kilómetros recorridos en un periodo de tiempo para brindar más información al algoritmo que calcula el 
scoring. 

Story Points: 2.

\item Como dueño quiero que la aplicación envíe información al servidor una vez al mes para hacer los cálculos del scoring. 

Story Points: 2.

\end{enumerate}

Decidimos asignarle 1 story point a la user story 1 dado que nos parece la más simple de realizar con respecto a otras, ya que no presenta dificultades técnicas ni de diseño. A partir de esta decisión, se estimaron el resto de las stories.

\section{Sprint}

Seleccionamos para el Sprint las user stories que son necesarias para la demo. 
Las mismas son las siguientes:
\begin{enumerate}
\item Como dueño quiero que la aplicación consulte un servicio de mapas para obtener información de velocidad máxima permitida del lugar transitado y si la zona es insegura o no. 

Story Points: 5.

\item Como dueño quiero registrar distintos tipos de eventos con sus costos asociados para sumar puntos al 
scoring. 

Story Points: 8.

\item Como dueño quiero poder agregar con facilidad nuevos tipos de eventos para hacer más flexible la 
aplicación. 

Story Points: 3.

\item Como dueño quiero que se procesen los datos recibidos para generar los scorings.

Story Points: 8.

\end{enumerate}

\section{Aclaraciones}
\begin{itemize}
\item La aplicación de celular registrará datos de la fecha/hora, posición y velocidad. Esta información se enviará al servidor y este se encargará de realizar los cálculos correspondientes para obtener el scoring.
\item El servidor será el encargado de conectarse con la aplicación de mapas para obtener la información de velocidades máximas y zonas inseguras.
\item Se asume que el celular del asegurado tiene prendido el bluetooth.
\end{itemize}
