\section{Product Backlog}

Las user stories del Backlog son:
\begin{enumerate}

\item Como dueño quiero que la aplicacion consulte un servicio de mapas para obtener informacion

Story points: 2

Criterio de aceptación:
Dado un par de coordenas se prueba si la información obtenida es la correcta.

Business Value: 5

\item Como dueño quiero que la aplicacion consulte un servicio de mapas para obtener informacion de velocidad máxima permitida del lugar transitado

Story points: 5

Criterio de aceptación:
Dado un par de coordenadas a las que le corresponda una velocidad máxima de 50 km/h, se consulta al servicio de mapas por la velocidad máxima y se obtiene 50 km/h.

Business value: 3

\item Como dueño quiero que la aplicacion consulte un servicio de mapas para obtener informacion de Zona insegura del lugar transitado

Story points: 5

Criterio de aceptación:
\begin{enumerate}
	\item Dado un par de coordenadas correspondientes a una zona segura, se consulta al servicio de mapas y la respuesta es que es zona segura.
	\item Dado un par de coordenadas correspondientes a una zona insegura, se consulta al servicio de mapas y la respuesta es que es zona insegura.
\end{enumerate}

Business value: 3

\item Como dueño quiero que el celular del asegurado se comunique por bluetooth con el dispositivo del auto para que la aplicacion reconozca que esta en el auto asegurado

Story points: 1

Criterio de aceptación:
Si me acerco con el celular al dispositivo colocado en el auto, se establece la conexión y se muestra una notificación de éxito. En caso de que falle se intentará cada 10 segundos.

Business value: 2


\item Como dueño quiero que la aplicación pueda conectarse en todo momento al auto del asegurado

Story points: 1

Criterio de aceptación:
Cuando se acerca el teléfono al dispositivo bluetooth se establece la conexión directamente, sin que el usuario tenga que abrir la aplicación

Business value: 3


\item Como dueño quiero que la aplicacion registre la información de la posición actual y velocidad para guardar los datos que servirán para los cálculos del scoring

Story points: 2

Criterio de aceptación:
Una vez establecida la conexión, la aplicación debe registrar cada 10 segundos la información.

Business value: 5

\item Como dueño quiero que la aplicacion registre el total de kilometros recorridos (en 1 mes) para brindar más información al algoritmo que calcula el scoring

Story points: 2

Criterio de aceptación:
Dada una lista de mediciones, que representan 10 kilometros recorridos, la aplicacion debe devolver 10 como cantidad de km recorridos, el cual es el valor esperado.

Business value: 1


\item Como dueño quiero que la aplicación envíe información al servidor una vez al mes para hacer los cálculos del scoring.

Story points: 2

Criterio de aceptación:
Se pueden enviar desde la aplicación los datos obtenidos y poder recibirlos desde el servidor.

Business value: 3

\end{enumerate}

\section{Sprint Backlog}

Las user stories del Sprint son:

\begin{enumerate}

\item Como dueño quiero que la aplicacion consulte un servicio de mapas ficticio para la demo para obtener informacion de velocidad máxima permitida del lugar transitado y si la zona es insegura o no.

Story points: 1

Criterio de aceptación:
Dado un par de coordenas se prueba si la información obtenida es la correcta.

Business Value: 5


\item Como dueño quiero registrar distintos tipos de eventos con sus costos asociados para sumar puntos al scoring

Story points: 3

Criterio de aceptación:
Los eventos solicitados hasta el momento son tenidos en cuenta por la aplicación

Business Value: 8


\item Como dueño quiero registrar eventos de cambio brusco

Story points: 5

Criterio de aceptación:
Dado un evento que suma 20 puntos de scoring por cambios bruscos en la velocidad:
\begin{enumerate} 
	\item Si no se registra ningun cambio brusco, no se sumaran puntos al scoring.
	\item Si se registra 1 cambio brusco, se sumaran 20 puntos al scoring.
	\item Si se registran 3 cambios bruscos, se sumaran 60 puntos al scoring.
\end{enumerate}

Business value: 5


\item Como dueño quiero registrar eventos de zona insegura

Story points: 3

Criterio de aceptación:
Dado un evento que suma 15 puntos de scoring si se transita una zona insegura:
\begin{enumerate}
	\item Si ninguna medicion fue registrada en zona insegura, no se sumaran puntos al scoring.
	\item Si 1 medicion fue registrada en zona insegura, se sumaran 15 puntos al scoring.
	\item Si 3 mediciones fueron registradas en zonas inseguras, se sumaran 45 puntos al scoring.
\end{enumerate}

Business value: 5


\item Como dueño quiero registrar eventos de velocidad maxima

Story points: 3

Criterio de aceptación:
Si tengo un evento que superando el 10% de la velocidad con un costo de 10 puntos de scoring:
\begin{enumerate}
	\item Si ninguna medicion supera por mas del 10% el limite de velocidad, no se sumaran puntos al scoring.
	\item Si 1 medicion supera por mas del 10% el limite de velocidad, se sumaran 10 puntos al scoring.
	\item Si 3 mediciones superan por mas del 10% el limite de velocidad, se sumaran 30 puntos al scoring. 
\end{enumerate}

Business value: 5


\item Como dueño quiero poder agregar con facilidad nuevos tipos de eventos para hacer más flexible la aplicacion

Story points: 3

Criterio de aceptación:
Poder agregar un nuevo tipo evento y se refleje en los cálculos del scoring

Business Value: 3


\item Como dueño quiero que se procesen los datos recibidos para generar los scorings.

Story points: 8

Criterio de aceptación:
Dado un conjunto de datos, se calcula el scoring a partir de los mismos.
\begin{enumerate}
	\item Con una medicion que exceda el limite de velocidad (con un evento con 10 puntos de scoring), debe devolver un scoring de 10.
	\item Con 2 mediciones que excedan el limite de velocidad (con un evento con 10 puntos de scoring), debe devolver un scoring de 20.
	\item Con 1 medicion que se encuentre en una zona insegura (con un evento con 30 puntos de scoring), debe devolver un scoring de 30.
\end{enumerate}

Business Value: 13


\end{enumerate}