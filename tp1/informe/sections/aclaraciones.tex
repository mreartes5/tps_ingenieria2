\section{Aclaraciones}

\begin{itemize}
\item La aplicación de celular registrará datos de la fecha/hora, posición y velocidad. Esta información se enviará al servidor y este se encargará de realizar los cálculos correspondientes para obtener el scoring.
\item El servidor será el encargado de conectarse con la aplicación de mapas para obtener la información de velocidades máximas y zonas inseguras.
\item Se asume que el celular del asegurado tiene prendido el bluetooth.
\item En el diagrama de clases no se muestran clases como Velocidad y Coordenada, ya que creemos que no son representativos al problema.
\item Asumimos que la unidad de la velocidad es siempre km/h.
\item La velocidad límite del evento de cambio brusco de velocidad no se define cuando se crea el mismo, sino que ya está predefinida.
\item Los eventos que conoce el Generador de Scoring están fijos a nivel código ya que para la demo no nos centramos en ese problema.
\end{itemize}