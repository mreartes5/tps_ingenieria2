\section{Comparaciones}

\subsection{Programming in the small vs Programming in the Large}

Cuando hablamos de \textit{Programming in the small} (DOO) es justamente como modelamos la 
realidad en una computadora. Utilizando objetos para representar entes de la 
realidad que colaboran entre ellos. Muchas veces podemos utilizar ciertos 
patrones, que son soluciones a problemas comunes de otros programadores al 
intentar modelar la realidad.
Sin embargo, existen algunos aspectos de la realidad que escapan a esta tecnica como 
son la concurrencia y disponibilidad que necesitan otras desiciones de diseño a 
un mas alto nivel para poder contemplar los atributos de calidad.
Esto es lo que se conoce como \textit{Programming in the Large} y es el foco principal del TP2,
donde tenemos que diseñar una solución.



\subsection{Comparación de los métodos utilizados}

A partir de la experiencia del grupo luego de haber desarrollado ambos trabajos prácticos se
pudieron encontrar las siguientes similitudes y diferencias. 


Una de las principales diferencias entre Unified Process (UP) y los métodos ágiles es que en la primera se sabe
desde el principio en qué tarea estará trabajando cada recurso en todo momento
del proyecto. En cambio, en la segunda en cada iteración se seleccionan qué tareas hacer y luego
cada integrante del grupo de trabajo las va tomando para desarrollarlas.


Otra diferencia se puede encontrar en las iteraciones de cada uno. En UP hay distintas
tipos de iteraciones (fases): Incio, Elaboración, Construcción y Transición. Cada una de las fases hace 
énfasis en distintas disciplinas. Por el contrario, en los métodos ágiles no hay distinción entre tipos 
de iteraciones.


Generalmente, en ambas técnicas, se utiliza una duración fija para las iteraciones de un mismo proyecto. 
Sin embargo, ésto no es una regla y la duración puede ser distinta en ciertas iteraciones.


En los métodos ágiles las tareas de cada iteración se escriben como user stories, mientras que en UP
se distinguen casos de uso que luego serán asignados a las iteraciones.


Otra disimilitud entre los métodos es que en UP se puede observar claramente las dependencias entre
tareas de una misma iteración, mientras que en los métodos ágiles, como Scrum, las tareas son independientes.
