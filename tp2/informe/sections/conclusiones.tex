\section{Conclusiones}

Por primera vez en la carrera tenemos que diseñar un sistema a gran escala considerando diversos
factores como la escalabilidad, disponibilidad, seguridad, etc. Como grupo nos entusiasmo diseñar una
aplicación con este enfoque más cercano a la industria, donde los escenarios son reales y
las desiciones son justificadas empíricamente. Por eso decimos que este TP se acerca más
a la realidad, donde los atributos de calidad como son la disponibilidad o performance se
ven comprometidos por la realidad misma: 

\begin{itemize}
  \item sismos
  \item baja conectividad
  \item alto volumen de datos
  \item ataques
  \item etc.  
\end{itemize}

El hecho de especificar los atributos de calidad y tenerlos presentes en todo momento nos
 ayudo a planificar y prever las falencias del sistema. Fuimos más conscientes de 
tomar desiciones de arquitectura y discutir diversas soluciones de antemano para 
evitar una mala planificación del proyecto.

Con la parte de planificación, la metodología ágil nos resulto bastante sencilla 
de realizar pues los cuatro estamos acostumbrados a utilizarla en el día a día 
laboral. Eso hizo que con las otras metodologías estemos un poco perdidos ya que 
ninguno tenia experiencia previa.