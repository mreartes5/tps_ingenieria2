\section{Especificación de Atributos de Calidad}

Escenario 1: Disponibilidad


Descripción: Ante eventualidades como pérdidas de información por catástrofes naturales, se desea que esta información se mantenga protegida y disponible en todo momento
\begin{itemize}
\item Fuente: Externa
\item Estímulo: Se corta la luz en Mendoza
\item Artefacto: Sistema
\item Entorno: Normal
\item Respuesta: Se redirigen los datos a procesar a otros nodos distribuidos que pueden completar el trabajo.  
\item Medición de respuesta: El 99,99\% de los casos la información se encuentra disponible. Además, los datos siguen estando disponible por los nodos replica.

\end{itemize} 

Escenario 2: Disponibilidad


Descripción: Se busca que haya conectividad en todo momento para que los datos de GPS se puedan enviar al sistema
\begin{itemize}
\item Fuente: Externa
\item Estímulo: Se detecta baja conectividad
\item Artefacto: Sistema
\item Entorno: Normal
\item Respuesta: Se notifica a los administradores del sistema y a un sistema alternativo para que provea conectividad.
\item Medición de respuesta: En menos de 12 horas llega el dron al lugar de baja conectividad.
\end{itemize} 

Escenario 3: Disponibilidad


Descripción: Cuando no hay buena conectividad se desea que los datos medidos por el GPS no se pierdan
\begin{itemize}
\item Fuente: Externa
\item Estímulo: No hay conectividad
\item Artefacto: Sistema de Controlador vehicular
\item Entorno: Normal
\item Respuesta: Se guardan los datos localmente hasta que vuelva a haber conectividad. 
\item Medición de respuesta: El 99,999\% de los datos no se pierden
\end{itemize} 
\textbf{PONER ALGO DEL TIEMPO MAX DE GUARDAR DATOS}


Escenario 4: Performance


Descripción: Se desea que el sistema ande muy rápido soportando el gran volumen de datos de todos los autos registrados del país.
\begin{itemize}
\item Fuente: Externa
\item Estímulo: Llegan 10000 mediciones hechas por GPS
\item Artefacto: Sistema
\item Entorno: Normal
\item Respuesta: Los datos son procesados y se generan las infracciones correspondientes
\item Medición de respuesta: En a lo sumo 5 segundos las infracciones son generadas satisfactoriamente.
\end{itemize} 

Escenario 5: Performance


Descripción: Los datos deben poder ser visualizados en un mapa en tiempo real, con el menor delay posible.
\begin{itemize}
\item Fuente: Externa
\item Estímulo: Se genera una medición en un vehículo que implica una infracción de exceso de velocidad
\item Artefacto: Sistema
\item Entorno: Normal
\item Respuesta: Se procesa la medición y se persiste la infracción.
\item Medición de respuesta: La información de la infracción se puede ver en el mapa en menos de 2 hs
\end{itemize} 

Escenario 6: Seguridad


Descripción: Es importante que el sistema esté protegido frente ataques externos
\begin{itemize}
\item Fuente: Individuo no identificado
\item Estímulo: Envío de mediciones fraudulentas
\item Artefacto: Sistema
\item Entorno: Normal
\item Respuesta: Se detecta que no proviene de una fuente confiable, se descarta y se loguea el ataque. 
\item Medición de respuesta: El 99,9\% de los casos se detecta el ataque satisfactoriamente.
\end{itemize} 

Escenario 7: Seguridad


Descripción: El acceso a datos está restringido a los roles de cada usuario.
\begin{itemize}
\item Fuente: Individuo identificado (empleado de Drones SA)
\item Estímulo: Consulta las mediciones recolectadas
\item Artefacto: Sistema
\item Entorno: Normal
\item Respuesta: Se brindan los datos a través de una interfaz web. 
\item Medición de respuesta: Los datos presentados corresponden al nivel de acceso que tiene el usuario autenticado.
\end{itemize} 

Escenario 8: Seguridad


Descripción: La sensibilidad de la información acumulada por nuestro sistema debe asegurarse de tal forma que permita auditar los movimientos en caso de ser pedido por la Defensoría del Pueblo.
\begin{itemize}
\item Fuente: Agente de la Defensoría del Pueblo
\item Estímulo: Desea auditar los movimientos de un conductor
\item Artefacto: Sistema
\item Entorno: Normal
\item Respuesta: Se muestra el registro detallado del último mes de actividad del conductor
\item Medición de respuesta: En el 99,99\% de los casos la información está disponible.
\end{itemize}
\textbf{EN LA PARTE DE ACLARACIONES PONER QUE NO SON REAL TIME, O DONDE SEA}

Escenario 9: Modificabilidad


Descripción: Se pretende la incorporación de nuevos tipos de infracciones.
\begin{itemize}
\item Fuente: Equipo de desarrollo
\item Estímulo: Se desea agregar un nuevo tipo de infracción
\item Artefacto: Sistema
\item Entorno: En ejecución
\item Respuesta: Se agrega un el nuevo tipo de infracción sin alterar otras funcionalidades 
\item Medición de respuesta: Se invierten menos de 8 horas hombre
\end{itemize} 

Escenario 10: Usabilidad
Descripción: Los usuarios deben tener una herramienta para poder acceder al historial de infracciones 
\begin{itemize}
\item Fuente: Externa
\item Estímulo: Un conductor pide revisar su actividad
\item Artefacto: Interfaz web para conductores 
\item Entorno: En diseño 
\item Respuesta: Se accede al repositorio de auditoría local y se muestran los datos solicitados al interesado
\item Medición de respuesta: El 95\% de los usuarios estuvo satisfecho con nivel 9. 
\end{itemize} 
\textbf{ACLARAR QUE SE LES PREGUNTA A 1000 USUARIOS EL NIVEL DE SATISFACCION DE 1 A 10}	
