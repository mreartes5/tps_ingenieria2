\section{Cronograma}

Se presentan las iteraciones de cada fase con los casos de uso correspondientes. 
También se muestra la duración de cada una de las iteraciones según las
horas hombre calculadas anteriormente para cada CU y las tareas previas.

\begin{itemize}

\item \textit{Primera iteración (Elaboración)}
\begin{itemize}
\item Enviando información del vehículo al servidor
\item Detectando baja conectividad
\end{itemize}

Duración: 2 semanas

\item \textit{Segunda iteración (Elaboración)}
\begin{itemize}
\item Avisando a empresa de drones
\item Consultando información de velocidades máximas y mínimas
\item Interactuando con el sistema de fotomultas
\end{itemize}

Duración: 1 semana

\item \textit{Tercera iteración (Construcción)}
\begin{itemize}
\item Recibiendo información del dispositivo del vehículo
\item Procesando/generando infracciones
\end{itemize}

Duración: 1 mes

\item \textit{Cuarta iteración (Construcción)}
\begin{itemize}
\item Guardando infracciones en el sistema 
\item Guardando información para estadísticas 
\item Guardando datos para la empresa de los drones 
\end{itemize}

Duración: 2 semanas

\item \textit{Quinta iteración (Construcción)}
\begin{itemize}
\item Actualizando puntos de los conductores 
\item Enviando valores al ministerio 
\end{itemize}

Duración: 2 semanas

\item \textit{Sexta iteración (Construcción)}
\begin{itemize}
\item Consultando historial de infracciones (según permisos) 
\item Accediendo a estadísticas de infracciones más frecuentes 
\item Consultando mapa de infracciones viales 
\end{itemize}

Duración: 3 semanas

\item \textit{Séptima iteración (Transición)}
\begin{itemize}
\item Haciendo deploy sistema en Ciudad de Buenos Aires 
\item Haciendo deploy sistema en La Pampa (baja conectividad) 
\end{itemize}

Duración: 2 mes

\item \textit{Octava iteración (Transición)}
\begin{itemize}
\item Haciendo deploy sistema en el resto del país
\end{itemize}

Duración: 3 mes

\end{itemize}

Los factores que se tuvieron en cuenta para decidir la cantidad de iteraciones y 
qué va en cada una de ellas fueron la funcionalidad y dependencias entre casos de uso.

La idea principal era tener varias iteraciones de cada fase de UP. En la primera 
iteración se pusieron los casos de uso riesgosos realizables en el tiempo 
establecido y que no tengan dependencias. Por otra parte, se tuvo en cuenta que era 
necesario tener una integración con los drones antes de los primeros cuatro meses.
